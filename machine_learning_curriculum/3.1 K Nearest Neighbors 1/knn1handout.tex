% You should title the file with a .tex extension (hw1.tex, for example)
\documentclass[11pt]{article}

\usepackage{amsmath}
\usepackage{amssymb}
\usepackage{fancyhdr}
\usepackage{amsthm}
\usepackage{mathtools}
\usepackage{ dsfont }
\usepackage{hyperref}
\usepackage{tikz}
\usepackage[]{algorithm2e}

\oddsidemargin-0.4cm
\topmargin-2cm     %I recommend adding these three lines to increase the 
\textwidth16.5cm   %amount of usable space on the page (and save trees)
\textheight23.5cm  

\newcommand{\question}[2] {\vspace{.25in} \hrule\vspace{0.5em}
\noindent{\bf #1: #2} \vspace{0.5em}
\hrule \vspace{.10in}}
\renewcommand{\part}[1] {\vspace{.10in} {\bf (#1)}}
\newtheorem{theorem}{Theorem}[section]
\newtheorem{lemma}[theorem]{Lemma}

\setlength{\parindent}{0pt}
\setlength{\parskip}{5pt plus 1pt}
 
\pagestyle{fancyplain}
\chead{\fancyplain{}{Introduction to Machine Learning}}

\begin{document}

\medskip                        % Skip a "medium" amount of space
                                % (latex determines what medium is)
                                % Also try: \Bigskip, \littleskip

\thispagestyle{plain}

\begin{center}                  % Center the following lines
{\Large Introduction to Machine Learning} \\
$K$-Nearest Neighbors Classification \\
March 20, 2018 \\
\end{center}

Today's activity will be done in groups of 2. The coding activity will be less structured than usual, so you'll have to rely a little more heavily on the pseudocode you write here!

\question{1}{Getting the $k$ nearest neighbors}

An ``instance'' of a star looks like this: \verb|[ageOfStarAtDeath, tempOfStarAtDeath, isSuperNova?]|

You're given a list of training instances (the ``training set") and one test instance, as well as a value $k$. Write pseudocode for the function \verb|getNeighbors| that takes these inputs and returns a list of the $k$ nearest training instances to the test instance. Some remarks that will be useful:
\begin{enumerate}
\item You can use a function \verb|distance(a,b)| that takes $a = (x_0,y_0)$ and $b = (x_1,y_1)$ and returns $dist(a,b)$.
\item We treat each star as a point on the Cartesian plane (i.e. $x$ = age of star at death, $y$ = temp of star at death).
\item If we sort a 2D list in Python, it will sort by the first element of each inner list. For example: \verb|sorted|($\big[[1,0],[0,1]\big]$) == $\big[[0,1],[1,0]\big]$ 
\end{enumerate}


\pagebreak

\question{2}{Getting the prediction}

Now that we have the $k$ nearest neighbors from above, write pseudocode to for the function \verb|getLabel| that gets the model's prediction on a test instance (i.e. the majority ``vote'' of the nearest neighbors to the instance). In this case, remember that there are only two possible classifications.

\vspace{6cm}

\question{3}{Test set performance}

Using the functions \verb|getNeighbors| and \verb|getLabel|, get the accuracy of a kNN model on a list of test instances (the ``test set"). The model's predictions depend on a training set and a given value of $k$. 


\vspace{6cm}

\question{4}{Write the code!}

Open \texttt{knnActivity.py} and do challenges 0-4! Make sure to refer to your pseudocode. 


\end{document}

